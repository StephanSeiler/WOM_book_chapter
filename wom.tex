\documentclass[letter,12pt]{article}

\usepackage[T1]{fontenc}
\usepackage[utf8]{inputenc}
\usepackage{lmodern}
\usepackage[english]{babel}
\usepackage{csquotes}
\usepackage{setspace}
\usepackage{fullpage}

\usepackage[
backend=biber,
style=authoryear,
natbib=true
]{biblatex}

\addbibresource{wom.bib}

\usepackage[]{hyperref}
\hypersetup{
    colorlinks=false,
}

\begin{document}
\title{Word of mouth}
\author{
	Stephan Seiler \\
	Stanford Graduate School of Business
	\and
	Song Yao \\
	Carlson School of Management, University of Minnesota
	\and
	Georgios Zervas \\
	Boston University Questrom School of Business
}
\maketitle

\doublespacing

\section{The Impact of Reviews on Consumer Demand and Firm Pricing}

Academic interest in consumer reviews goes nearly as far back as the
appearance of the first online reputation systems on e-commerce platforms like
eBay and Amazon. Reviews -- the information units of reputation systems --
typically consist of a numeric score (typically represented as a rating
between 1 and 5 stars), the review submission date, and some open-ended text
that allows consumers to evaluate a business (or a product) in their own
words. Some reputation systems allow consumers to attach pictures to their
reviews, and to separately rate businesses on dimensions such as service and
price. Another commonly implemented feature lets consumers rate others'
reviews as ``helpful'' or ``useful''. Reputation systems also display certain
basic facts about reviewers such as their review history, a profile picture,
and location, though they do not require that reviewers disclose their true
identity. Most reputation systems allow anyone to submit to review, which has
raised concerns about review
fraud~\citep{mayzlin2014promotional,luca2016fake}. Various strategies are in
place to mitigate this concern including publishing reviews only by consumers
whose purchases can be verified, highlighting reviews that can be linked to
verified purchases using special badges, and relying on fraud-detection
algorithms to discard fake reviews.

\subsection{Review valence}

The most common way in which reputation systems aggregate the plethora of
information they collect is to compute an average rating for each
business.\footnote{An average rating is simply the \emph{unweighted} mean of
all individual ratings a business or product has received, often rounded to
the nearest decimal point of half-star.} Average ratings, which are meant to
capture overall quality, are prominently displayed and used to rank products
and businesses in response to user queries. For instance, the query ``hotels
in San Francisco'' on TripAdvisor is likely to return higher-rated hotels as
the top search results. 

Maybe due to their simplicity, average ratings have been well-studied. By now,
it is well-established that average ratings have a substantial causal impact
on demand and pricing, though the magnitude of these effect varies by the
timing and context of the study.

eBay's reputation system, which allows buyers and sellers on the platform to
rate each other, was among the first to be studied and has been the focus of
tens of studies~\citep{ba2002evidence,houser2006reputation,lucking2007pennies,eaton2002value,bajari2003winner,kalyanam2001return,mcdonald2002reputation,cabral2010dynamics,dewally2006reputation,jin2006price}. 
The findings of these papers, which rely on different methods and study
different eBay product categories are broadly consistent: highly-rated sellers
attract more bidders in their auctions, fetch higher prices, and sell their
items with higher probability. Interestingly, \citet{jin2006price} show that
the while high-rated sellers can charge a premium, this is not because they
sell higher quality products.

Similar effects for review valence have been demonstrated on other review and
e-commerce platforms. For example, \citet{chevalier2006effect} find that the
sales ranks of books on Amazon depend on their average ratings. Looking at
Yelp, \citet{luca2016reviews} and \citet{anderson2012learning} estimate the
impact of average ratings on restaurant demand and respectively find that a
one-star increase in ratings causes a 5\% increase in revenue and a 50\%
increase in the probability of being sold out. Both of these studies rely on a
clever regression discontinuity (RD) design that exploits the fact that Yelp
rounds average ratings to the nearest half-star: a business with a 4.24
average is rounded down to 4 stars, while a business with a nearly identical
rating of 4.25 stars is up to 4.5 stars. Using the same RD strategy
\citet{luca2013digitizing} study ZocDoc, a platform that allows consumers to
rate doctors and make appointments with them, and find that a half-star
increase in ratings is associated with 10\% increase that an appointment will
be filled.

While many studies have investigated the impact of average ratings, less is
known about other aggregate measures. One exception is the work of
\citet{sun2012variance} who studies the variance of ratings. Reputation
systems do not explicitly display rating variances. However, most display
histograms of reviews counts broken down by rating. By looking at these
histograms consumers can infer rating variance. Motivated by a theoretical
model, \citet{sun2012variance} presents empirical evidence that increased
variance in ratings has a positive impact on the demand of low-rated products.
The key intuition behind this result is that low-rated products with some high
individual ratings (\emph{i.e.}, high variance) may be appealing to certain
consumers, while products with consistently low ratings (\emph{i.e.}, low
variance) are less likely to be appealing to anyone.

A number of papers have examined the moderators of the relationship between
ratings and demand. A common theme that has emerged is that the relationship
between ratings and demand (or prices) depends on consumers' prior quality
uncertainty. For instance, \citet{luca2016reviews} finds that that Yelp
ratings do not matter for chains because there exists little a priori
uncertainty about their quality. Similarly, \citet{lewis2016welfare} find that
the impact of TripAdvisor ratings is much smaller for chain-affiliated hotels
than it is for independently operated properties.

The effects we have discussed above are unlikely to represent long-term
equilibrium outcomes. As consumers pay more attention to the increasing number
of reviews accumulating online, the influence of these reviews on demand and
pricing will also likely become stronger. \cite{lewis2016welfare} provide some
evidence for the evolution of these treatment effect by looking at hotel
demand as a function of TripAdvisor ratings over a decade. The study finds
that the impact of a 1-star increase in a hotel's average rating went for zero
to 25\% between 2004 and 2014.

In conclusion, we point out that the existence of a large number of studies of
ratings across different contexts has enabled meta-analyses of their
effects~\citep{floyd2014online,babic2016effect}. By comparing treatments
effects estimated in different settings, meta-analysis can reveal interesting
moderators the effect of ratings on demand. For instance,
\cite{babic2016effect} find that early reviews are more important for new
products than they are for new services.

\subsection{Review volume}

Another salient and well-studied feature of reputation systems is review
volume. Review counts are typically displayed prominently next to average
ratings, and consumers can also use them to infer quality. All else equal, we
may expect that consumers will have less uncertainty about the quality of
products with more reviews. Testing the hypothesis that changes in review
volume affect demand raises a unique identification challenge: while more
reviews may cause sales, the reverse is also true, \emph{i.e.}, sales can
result in more reviews. Thus, to avoid this reserve causality issue, reliable
estimates of review volume on sales have primarily relied on experimental
methods.

Two experimental studies investigate the link between review volume and
demand. \citet{resnick2006value} compare new eBay sellers with sellers that
have an established reputation and find consumers' willing to pay is 8\%
higher for the latter. \citet{pallais2014inefficient} studies oDesk, an online
marketplace where employers can hire workers, and finds that workers who
(randomly) received detailed feedback had better employment outcomes,
including higher wages and earnings. An interesting implication of these
studies is that having at establishing a reputation by having at least one
review matters more than receiving an additional review.

\subsection{Review text}

Review text is inherently high-dimensional and typically not amenable to
analysis by the same methods researchers use to study variables like review
valence and volume. Therefore, most analyses of review text rely on a
pre-processing step that transforms text into a small number of variables that
capture variation along dimensions of interest. These variables can
subsequently be analyzed using traditional econometric approaches.

At a high level, two pre-processing approaches have been employed in the
literature to convert text into a manageable number of variables. First,
researchers have used statistical approaches that rely on dimensionality
reduction algorithms to map text onto low dimensional measures. Statistical
approaches themselves vary in their sophistication from the direct application
of simple formulas to compute metrics like readability to complex methods like
topic modeling that can extract latent structure and meaning from unstructured
data. \citet{moe2017social} provide a comprehensive review of the statistical
approaches used to analyze text from reputation systems and social media. The
second approach is grounded in consumer behavior theory and starts by forming
a hypothesis regarding the effects of specific text constructs. The presence
of constructs in text is often detected manually (\emph{e.g.}, using human
coders), or via simple pattern matching rules.

\citet{buschken2016sentence}. 
Write a paragraph about text. Cite \citet{ghose2012designing},
\citet{packard2017language}.

\citet{ghose2011estimating} should be cited too.

\citet{ghose2012designing} use text-mining study various aspects of review
text on TripAdvisor, finding that simpler and shorter reviews that do not
contain spelling errors have a positive impact on hotel demand.

\citet{packard2017language} combine lab experiments and field data to
investigate how the language consumers use in to endorse products affects how
persuasive their reviews are. The authors find explicit endorsements (``I
recommended this book'') are more persuasive than implicit endorsements (``I
enjoyed this book''). Interestingly, the authors find that in practice novice
users are more likely to use explicit endorsements that experts, suggesting
that novice reviews are more persuasive.

\citet{kupor2017spontaneous} also use lab experiments to investigate the
language the endorsements, focusing on the effect of endorsement authenticity.
The author)s find that endorsements that perceived to be more authentic are
more convincing that endorsements that are more thoughtful. For example,
tweets that are less polished or contain typographical errors are perceived as
more authentic, thus enhancing their persuasiveness.

\section{Other Literature Reviews}

\textbf{GZ: This para may be better suited for the overall chapter conclusion
than this particular section.} Discuss \citet{dellarocas2003digitization} and
\citet{luca2015user}.

\printbibliography

\end{document}