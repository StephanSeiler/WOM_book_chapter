\documentclass[letter,12pt]{article}

\usepackage[T1]{fontenc}
\usepackage[utf8]{inputenc}
\usepackage{lmodern}
\usepackage[english]{babel}
\usepackage{csquotes}
\usepackage{setspace}
\usepackage{fullpage}

\usepackage[
backend=biber,
style=authoryear,
natbib=true
]{biblatex}

\addbibresource{wom.bib}

\usepackage[]{hyperref}
\hypersetup{
    colorlinks=false,
}

\begin{document}
\title{Word of mouth}
\author{
	Stephan Seiler \\
	Stanford Graduate School of Business
	\and
	Song Yao \\
	Carlson School of Management, University of Minnesota
	\and
	Georgios Zervas \\
	Boston University Questrom School of Business
}
\maketitle

\doublespacing

\section{The Impact of Reviews on Consumer Demand and Firm Pricing}

Academic interest in consumer reviews goes nearly as far back as the
appearance of the first online reputation systems on e-commerce platforms like
eBay and Amazon. Reviews -- the information units of reputation systems --
typically consist of a numeric score (typically represented as a rating
between 1 and 5 stars), the review submission date, and some open-ended text
that allows consumers to evaluate a business in their own words. Some
reputation systems allow consumers to attach pictures to their reviews, and to
separately rate businesses on dimensions such as service and price. Another
commonly implemented feature lets consumers rate others' reviews as
``helpful'' or ``useful''. Reputation systems display certain basic facts
about reviewers such as their review history, a profile picture, and location,
though they do not require that reviewers disclose their true identity. Most
reputation systems allow anyone to submit to review, which has raised concerns
about review fraud~\citep{mayzlin2014promotional,luca2016fake}. Various
strategies are in place to mitigate this concern including publishing reviews
only by consumers whose purchases can be verified, highlighting reviews that
can be linked to verified purchases using special badges, and relying on
fraud-detection algorithms to discard fake reviews.

\subsection{Review valence}

The most common method of aggregating the plethora of information reputation
systems collect is to compute an average rating for each business.\footnote{An
average rating is simply the \emph{unweighted} mean of all individual ratings
a business or product has received, often rounded to the nearest decimal point
of half-star.} Average ratings, which are meant to capture overall quality,
are prominently displayed and used to rank products and businesses in response
to user queries. For instance, the query ``hotels in San Francisco'' on
TripAdvisor is likely to return higher-rated hotels as the top search results.
Maybe due to their simplicity, average ratings have been well-studied. By now,
it is well-established that average ratings have a substantial causal impact
on demand and pricing, though the magnitude of these effect varies by the
timing and context of the study.

eBay's reputation system, which allows buyers and sellers on the platform to
rate each other, was among the first to be studied and has been the focus of
tens of studies~\citep{ba2002evidence,houser2006reputation,lucking2007pennies,eaton2002value,bajari2003winner,kalyanam2001return,mcdonald2002reputation,cabral2010dynamics,dewally2006reputation,jin2006price}.
The findings of these papers, which rely on different methods and study
different eBay product categories are broadly consistent: highly-rated sellers
attract more bidders in their auctions, fetch higher prices, and sell their
items with higher probability. Interestingly, \citet{jin2006price} show that
the while high-rated sellers can charge a premium, this is not because they
sell higher quality products.

Similar effects for review valence have been demonstrated on other review and
e-commerce platforms. \citet{chevalier2006effect} finds that the sales ranks
of books on Amazon depend on their average ratings. For example, looking at
Yelp, \citet{luca2016reviews} and \citet{anderson2012learning} estimate the
impact of average ratings on restaurant demand and respectively find that a
one-star increase in ratings causes a 5\% increase in revenue and a 50\%
increase in the probability of being sold out. \citet{luca2013digitizing}
study ZocDoc, a platform that allows consumers to rate doctors and make
appointments, and find a half-star increase in ratings is associated with 10\%
increase that an appointment will be filled.

While many studies have investigated the impact of average ratings, less is
known about other aggregate measures. One exception is the work of
\citet{sun2012variance} who studies the variance of ratings. While reputation
systems do not explicitly display rating variances, many of them display
percentages of reviews by rating from which consumers can infer variance.
\citet{sun2012variance} presents theoretical and empirical evidence that
increased variance in ratings has a positive impact on the demand of low-rated
products. The key intuition behind this result is that low-rated products with
some high individual ratings (\emph{i.e.}, high variance) may be appealing to
certain consumers, while products with consistently low ratings (\emph{i.e.},
low variance) are less likely to be appealing to anyone.

A number of papers have examined the moderators of the relationship between
ratings and demand. A common theme that has emerged is that the relationship
between ratings and demand (or prices) depends on consumers' prior quality
uncertainty. For instance, \citet{luca2016reviews} finds that that Yelp
ratings do not matter for chains presumably because there exists little a
priori uncertainty about their quality. Similarly, \citet{lewis2016welfare}
find that the impact of TripAdvisor ratings is much smaller for
chain-affiliated hotels than it is for independently operated properties.

Finally, it is worth noting that the effects of consumer reviews on demand and
pricing discussed thus far are unlikely to represent long-term equilibrium
outcomes. As reviews accumulate online, and as consumers pay more attention to
these reviews, their influence on demand and pricing is also likely become
stronger. A recent paper by \cite{lewis2016welfare} documents this evolution
by looking at hotel demand as a function of reviews over a decade. The study
finds that the impact of a 1-star increase in a hotel's average rating went
for zero to 25\% between 2004 and 2014.

\subsection{Review volume}

Another salient and well-studied feature of reputation system is the review
volume. Review counts are typically displayed prominently next to ratings, and
consumers can also use them to infer quality. For instance, all else equal, we
may expect that consumers will less uncertainty about the quality of products
with more reviews. 

Two experimental studies find support for this hypothesis.
\citet{resnick2006value} compare new eBay sellers with sellers that have an
established reputation and find consumers' willing to pay is 8\% higher for
the latter. \citet{pallais2014inefficient} studies oDesk, an online
marketplace where employers can hire workers, and finds that workers who
(randomly) received detailed feedback had better employment outcomes,
including higher wages and earnings.

\subsection{Review text}

Write a paragraph about text. Cite \citet{ghose2012designing},
\citet{packard2017language}.

\citet{ghose2012designing} use text-mining study various aspects of review
text on TripAdvisor, finding that simpler and shorter reviews that do not
contain spelling errors have a positive impact on hotel demand.

\section{Meta-analyses of reputation systems}

A large number of studies that vary in their timing, setting, and methods have
been conducted to measure the effects of reputation on various economic
outcomes. One way to summarize these effects is via meta-analyses. Write a
paragraph about meta-analyses like \citet{babic2016effect} and
\citet{floyd2014online}.

\section{Other Literature Reviews}

Discuss \citet{dellarocas2003digitization} and \citet{luca2015user}.

\section{Discussion etc}

Blah blah

\printbibliography

\end{document}